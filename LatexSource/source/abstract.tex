% !TeX spellcheck = en_US

%
%
\thispagestyle{empty}\vspace*{2cm}%
\section*{Abstract}%

    %Over the last few years, the electrical and/or electronic architecture of vehicles has been significantly developing. The new generation of vehicles demands a considerable amount of computational power due to a large number of safety-critical applications and ADAS functionalities. Accordingly, a high-performance computing unit is needed to provide the required power and process the applications while, in this case, vehicle architecture moves toward a centralized architecture. Simultaneously, appropriate software architecture has to be specified to satisfy the needs of the main computing unit and functional safety requirements. Furthermore, because of the significant number of sensors and actuators, high-bandwidth protocols are required. Furthermore, to be able to integrate safety and real-time critical applications, in other words, mixed-criticality systems, being deterministic and redundant in these protocols are prerequisites. However, the process of configuring and integrating critical applications into a vehicle's central computer while meeting safety requirements, and also guaranteeing reliable communication as considering optimization objectives are time-consuming, complex, and error-prone tasks. In this paper, we present a novel model-based framework, \textit{E/E Designer}, to facilitate synthesis of car E/E architecture supporting automotive embedded systems modeling, automatic mapping process of software components to hardware elements satisfying safety requirements such as thread scheduling, creation of network message routing and system scheduling for car's topology meeting safety demands e.g., redundancy, and optimization of the system model applying multi-objective optimization. In addition, this framework utilizes a single-step approach to solve ILP constraints in order to reduce the solving time and consider the relations among various constraints. Finally, we evaluate the framework performance by proposing a design-time and an experimental setup evaluations.

    %Over the last few years, the electrical and/or electronic (E/E) architecture of vehicles has significantly developed. The new generation of road vehicles demands considerable computational power due to many safety-critical applications and advanced driver assistance systems (ADAS) functionalities. A centralized architecture with the adoption of a high-performance computing unit establishes a proper way in empowering vehicles to process the demanding applications. In addition, high-bandwidth protocols are required due to the significant number of sensors and actuators. Moreover, deterministic and redundancy protocols are necessary to integrate safety and real-time critical applications called as mixed-criticality systems. 
    
    In recent years, the field of automotive electrical and electronic (E/E) architecture has undergone substantial evolution. The latest generation of road vehicles necessitates a substantial infusion of computational power to support the execution of a multitude of safety-critical applications and advanced driver assistance systems (ADAS) functionalities. Centralized architecture, enhanced by the incorporation of high-performance computing units, emerges as a pivotal approach to reinforce the capability of vehicles in handling these resource-intensive applications. The high number of sensors and actuators demands high-bandwidth communication protocols to facilitate a seamless data flow. In addition, to harmonize the integration of safety-critical and real-time applications, known as mixed-criticality systems, deterministic and redundancy protocols are necessary.
    However, configuring and integrating essential applications into a vehicle's E/E architecture, all while meeting various requirements, guaranteeing reliable communication, and considering optimization objectives, can be time-consuming, complex, and error-prone tasks. 
    
    
    %This thesis presents a novel model-based framework to facilitate the synthesis of a car's E/E architecture supporting automotive embedded systems modeling. The introduced tool automates mapping of software components to hardware elements and computes schedules for application threads. It establishes network message routing and schedules communication tasks within the car's topology while addressing safety requirements such as redundancy, homogeneous redundancy, and reliability. The proposed computer-aided tool also optimizes the system model covering multiple optimization objectives. It also supports multi-objective optimization, and utilizes a single-step approach to solve mixed-integer programming (MIP) constraints in order to reduce the solving time and consider the relations among various constraints.
    
    
    This thesis presents a novel model-based framework to facilitate the synthesis of car E/E architectures, which supports modeling for automotive embedded systems. The introduced tool automates mapping of software components to hardware elements and computes schedules for application threads. It establishes network message routing and schedules communication tasks within the car's topology while addressing safety requirements, including redundancy, homogeneous redundancy, and reliability.
    The proposed computer-aided tool also optimizes the system model, covering multiple optimization objectives. It supports multi-objective optimization and utilizes a single-step approach to solve mixed-integer programming (MIP) constraints, reducing solving time and considering the relationships among various constraints. Moreover, the proposed tool offers a web-based frontend that allows users to model their desired E/E systems and select various hardware and software requirements and properties, along with the boundary and optimization goals. The developed frontend also visualizes the solution of the designed system after it has been solved.
    
    %This thesis presents a novel model-based framework to facilitate the synthesis of a car's E/E architecture supporting automotive embedded systems modeling. This framework includes an automatic mapping process of software components to hardware elements that satisfies safety requirements, such as application thread scheduling. This framework includes an automatic mapping process of software components to hardware elements that satisfies safety requirements, such as application thread scheduling. It creates network message routing and communication task scheduling for the car's topology, meeting safety demands such as redundancy. The framework also optimizes the system model using multi-objective optimization, and utilizes a single-step approach to solve mixed-integer programming (MIP) constraints in order to reduce the solving time and consider the relations among various constraints. 
    
    %Design error analysis approach:
    %To address this question, an approach, called design error analysis, is introduced, as depicted in Chapter~\ref{designerror}. 
    
    There are situations where a designed E/E architecture is not satisfiable, meaning that feasible solutions cannot be found by a MIP solver. Unlike simple models, navigating and correcting the unsatisfiability of complex E/E models is a complex and time-consuming task, leading to increased development costs. To tackle this issue, this thesis also introduces an approach to identify design errors when violations occur in a constraint set included in a system model after the solving step. %This feature is crucial for detecting and rectifying errors in the system design within a reasonable timeframe, ensuring that the system is optimized and meets all necessary constraints and requirements.
    
     The performance of this model-based framework is assessed at three key stages. Design-time, where the solving and generation times of constraint sets in various scenarios are evaluated, including scalability analysis. Run-time, where the solution is deployed on an experimental setup, and finally, quantitative and qualitative evaluations. The results of the design-time experiments indicate that the formulations can scale to systems of reasonable size. During the run-time experiments, it is observed that there are no instances of timing deadline breaches following the deployment of the design-time solutions on an experimental setup.
     
    %The performance of the model-based tool is assessed using three methods: a design-time evaluation, where the solving and generation times of constraint sets in different scenarios are evaluated, including scalability analysis; a run-time evaluation, where the solution is deployed on an experimental setup; and quantitative and qualitative evaluation, where the performance, usability, and practicality of the E/E Designer are assessed by addressing a series of use cases through both manual and automated approaches.
    %The design-time experiments show that our formulations scale to systems with reasonably large sizes. During the run-time experiments, it was noted that there were no instances of timing deadline breaches following the deployment of the design-time solutions on an experimental setup.



    %In recent years, the electrical and/or electronic architecture of vehicles has been significantly evolving. The new generation of cars demands a considerable amount of computational power due to a large number of safety-critical applications and driver-assisted functionalities. Consequently, a high-performance computing unit is required to provide the demanded power and process these applications while in this case,  vehicle architecture moves toward a centralized architecture.
    %\mynotes{and why not distributed? (maybe there are no significant reasons...just a note from me)}.Simultaneously, appropriate software architecture has to be defined to fulfill the needs of the main computing unit and functional safety requirements. However, the process of configuring and integrating critical applications into a vehicle central computer while meeting safety requirements and optimization objectives is a time-consuming, complicated and error-prone process. In this paper, we firstly present the evolution of the vehicle architecture, past, present, and future, and its current bottlenecks and future key technologies. Then, challenges of software configuration and mapping for automotive systems are discussed. Accordingly, mapping techniques and optimization objectives for mapping tasks to multi-core processors using design space exploration method are studied. Moreover, the current technologies and frameworks regarding the vehicle architecture synthesis, model analysis with regard to software integration and configuration, and solving the mapping problem for automotive embedded systems are expressed. Finally, we propose four research questions as future works for this field of study.



 %In recent years, the electrical and/or electronic (E/E) architecture of vehicles has evolved significantly, driven by the demand for increased computational power to support safety-critical applications and advanced driver assistance systems (ADAS) functionalities. This evolution has led to the adoption of centralized architectures with high-performance computing units. These architectures also require high-bandwidth and deterministic communication protocols to handle numerous sensors and actuators, especially in mixed-criticality systems. %However, configuring and integrating these components while ensuring safety, reliability, and optimization can be complex and time-consuming.
 
 %However, configuring and integrating essential applications into a vehicle's E/E architecture while meeting safety requirements, guaranteeing reliable communication, and considering optimization objectives are time-consuming, complex, and error-prone tasks.
 %This paper presents a novel model-based framework, called \textit{E/E Designer}, to facilitate the synthesis of a car's E/E architecture supporting automotive embedded systems modeling. This framework includes an automatic mapping process of software components to hardware elements that satisfies safety requirements, such as application thread scheduling. The framework automates mapping of software components to hardware elements and computes schedules for application threads.
%This framework includes an automatic mapping process of software components to hardware elements and calculating schedules for application threads. It establishes network message routing and schedules communication tasks within the car's topology while addressing safety requirements such as redundancy. The \textit{E/E Designer} also optimizes the system model using multi-objective optimization, and utilizes a single-step approach to solve mixed-integer programming (MIP) constraints in order to reduce the solving time and consider the relations among various constraints. We use an experimental setup to investigate the framework's performance through design-time and run-time evaluations. The results of our design-time experiments indicate that our formulations can scale to systems of reasonable size. During our run-time evaluations, we observed no timing deadline violations after deploying the design-time solutions on the experimental setup. In our run-time experiments, no timing deadline violations happened after the deployment of the design-time solutions, created by the framework, on an experimental setup. 



%
%

        \newpage%
         \thispagestyle{empty}\vspace*{2cm}%
\section*{Zusammenfassung}%
       In den letzten Jahren hat sich der Bereich der elektrischen und elektronischen (E/E-) Architektur von Kraftfahrzeugen erheblich weiterentwickelt. Die neueste Generation von Straßenfahrzeugen erfordert eine erhebliche Steigerung der Rechenleistung, um die Ausführung einer Vielzahl von sicherheitskritischen Anwendungen und fortschrittlichen Fahrerassistenz-systemen (ADAS) zu unterstützen. Eine zentralisierte Architektur, die durch den Einbau von Hochleistungsrecheneinheiten verbessert wird, erweist sich als entscheidender Ansatz, um die Fähigkeit von Fahrzeugen zur Handhabung dieser ressourcenintensiven Anwendungen zu verbessern. Darüber hinaus erfordert die zunehmende Anzahl von Sensoren und Aktoren Kommunikationsprotokolle mit hoher Bandbreite, um einen nahtlosen Datenfluss zu ermöglichen. Zur Harmonisierung der Integration von sicherheitskritischen und Echtzeitanwendungen, die als Systeme mit gemischter Kritikalität bezeichnet werden, sind außerdem deterministische und redundante Protokolle unerlässlich.
        Die Konfiguration und Integration wichtiger Anwendungen in die E/E-Architektur eines Fahrzeugs unter Berücksichtigung der verschiedenen Anforderungen, der Gewährleistung einer zuverlässigen Kommunikation und der Optimierungsziele kann jedoch zeitaufwändig, komplex und fehleranfällig sein. 

        In dieser Arbeit wird ein neuartiges modellbasiertes Framework zur Erleichterung der Synthese der E/E-Architektur eines Fahrzeugs vorgestellt, das die Modellierung für eingebettete Systeme im Automobil unterstützt. Das vorgestellte Werkzeug automatisiert die Zuordnung von Softwarekomponenten zu Hardwareelementen und berechnet Zeitpläne für Anwendungsthreads. Es legt das Routing von Netzwerknachrichten fest und plant Kommunikationsaufgaben innerhalb der Fahrzeugtopologie unter Berücksichtigung von Sicherheitsanforderungen wie Redundanz, homogener Redundanz und Zuverlässigkeit.
        Das vorgeschlagene computergestützte Tool optimiert auch das Systemmodell und deckt mehrere Optimierungsziele ab. Es unterstützt die Mehrzieloptimierung und verwendet einen Ein-Schritt-Ansatz zur Lösung von MIP-Beschränkungen (Mixed-Integer Programming), wodurch die Lösungszeit reduziert und die Beziehungen zwischen den verschiedenen Beschränkungen berücksichtigt werden. Darüber hinaus bietet das vorgeschlagene Tool ein webbasiertes Frontend, mit dem Benutzer ihre gewünschten E/E-Systeme modellieren und verschiedene Hardware- und Softwareanforderungen und -eigenschaften sowie die Randbedingungen und Optimierungsziele auswählen können. Das entwickelte Frontend visualisiert auch die Lösung des entworfenen Systems, nachdem es gelöst wurde.
%
        Es gibt Situationen, in denen eine entworfene E/E-Architektur nicht zufriedenstellend ist, was bedeutet, dass der MIP-Löser keine machbaren Lösungen finden kann. Im Gegensatz zu einfachen Modellen ist das Navigieren und Korrigieren der Unerfüllbarkeit komplexer E/E-Modelle eine komplexe und zeitaufwändige Aufgabe, die zu erhöhten Entwicklungskosten führt. Um dieses Problem anzugehen, wird in dieser Arbeit auch ein Ansatz zur Identifizierung von Entwurfsfehlern vorgestellt, wenn nach dem Lösungsschritt Verletzungen in der im Systemmodell enthaltenen Constraintmenge auftreten.
    
        Die Leistung dieses modellbasierten Rahmens wird in drei Schlüsselphasen bewertet: zur Entwurfszeit, in der die Lösungs- und Generierungszeiten von Constraint-Sets in verschiedenen Szenarien bewertet werden, einschließlich einer Skalierbarkeitsanalyse; zur Laufzeit, in der die Lösung in einem Versuchsaufbau eingesetzt wird, sowie durch quantitative und qualitative Bewertungen. Die Ergebnisse der Experimente zur Entwurfszeit zeigen, dass die Formulierungen auf Systeme angemessener Größe skaliert werden können. Bei den Laufzeit-experimenten wurde festgestellt, dass nach dem Einsatz der Lösungen aus der Entwurfszeit in einem Versuchsaufbau keine Verstöße gegen die Zeitvorgaben auftraten.

        %\thispagestyle{empty}\vspace*{2cm}%
        \newpage
        \thispagestyle{empty}\vspace*{2cm}%

        \section*{{Acknowledgement}}%
%\blindtext%

        I extend my deepest gratitude to all those who have contributed to the realization of this Ph.D. thesis.
        First and foremost, I express my sincere appreciation to my supervisor, Professor Alois Knoll, for his unwavering support, mentorship, and invaluable guidance throughout this research journey and for giving me the chance to pursue my Ph.D. at his research group. His expertise and commitment have been instrumental in shaping the trajectory of my academic pursuits. I would also like to express my heartfelt thanks to Professor Ali Mosleh, my second examiner from University of California, Los Angeles (UCLA), for his valuable feedback and insightful comments on my research. His expertise has been instrumental in refining the quality of my work. Additionally, my sincere appreciation goes to Professor Jörg Ott, the chairman of my defense, for overseeing the defense process and for his time and support in ensuring everything runs smoothly.

        
        A special acknowledgment is reserved for Dr. Morteza Hashemi Farzaneh as my mentor. His guidance and insightful hints provided in the early stages were pivotal in helping me find my research direction.
        I wish to express my gratitude to my colleagues at the Chair of Robotics, Artificial Intelligence, and Real-time Systems. In particular, I extend my special thanks to Amy Buecherl for her consistent help and kindness. A heartfelt thank you goes to Ute Lomp, the former member of the chair, for her unwavering support and care. Welcoming Janine Delle, a new member of the chair, deserves my appreciation for her kindness and contributions to the dissertation process. I also want to thank Marie-Luise Neitz for her valuable assistance. Dr. Alex Lenz merits special thanks for his unwavering support, help, and kindness throughout this journey. My sincere gratitude extends to Thilo Mueller, my student, for his contributions. I am indebted to my colleagues and friends, including Hossein Malmir, Walter Zimmer, and Soubarna Banik, for their proofreading of this thesis. A profound thank you goes to my best friend, Andreas Wieser, for his help and for being an exceptional companion and friend during this journey.

        In conclusion, I express deep gratitude to my family, including my sister and brothers, for their continuous support, motivation, and assistance. To my parents, Fatemeh and Yousef, my heartfelt and special thanks for their unwavering love, motivation, and constant support throughout this challenging journey. Despite the long distance, my parents and family send their love and support. I am immensely grateful and love you all dearly. This work is dedicated to them.



        
        %I am profoundly thankful to the members of my doctoral committee, [Committee Members' Names], for their insightful feedback and constructive critique, which significantly enriched the quality of this work.
        
        %A heartfelt thank you goes to my family for their unwavering encouragement, understanding, and love. Their constant support has been my pillar of strength.I am indebted to my friends and colleagues for their camaraderie, shared ideas, and encouragement during both the challenging and rewarding phases of this endeavor. Their friendship has made this academic pursuit all the more meaningful.
        
        %I also acknowledge the financial support provided by [Funding Agency/Organization], which enabled the successful completion of this research.Special thanks to the staff at [Department/Institution] for their administrative support and facilitation of resources throughout my academic journey.This work is dedicated to [optional dedication].In summary, to everyone who has been part of this incredible journey – your contributions, support, and encouragement have played a vital role in making this Ph.D. thesis a reality. Thank you.
        
        
        
        \vfill%
        %\RAIaddressCityChair
        Los Angeles, \RAIutilsDate{14}.{12}.{2023} \hfill\textit{Hadi Askaripoor}%
        \vspace*{2cm}%
        %\null
        %\thispagestyle{empty}
        %\afterpage{\null\newpage\addtocounter{page}{1}}
        %\leavevmode\thispagestyle{empty}\newpage
